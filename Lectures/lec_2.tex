\chapter{Overview of EM}
\lecture{2}{Jan 27 10:10}{}

\begin{remark}
    Differential Calculus: Gradient, Divergence 

    \begin{definition}
        Del operator

        \[
            \nabla = i_x \frac{\partial }{\partial x} + i_y \frac{\partial }{\partial y}  
            + i_z\frac{\partial }{\partial z} 
        \]
        The del operator alone is meaningless, however when used with scalar or vector fields,
        it gives rise to interesting concepts
    \end{definition}

    \begin{definition}
        We define Divergence as 
    \[
        \nabla \cdot \vec{A} = \frac{\partial A_x}{\partial x} + \frac{\partial A_y}{\partial y}
        + \frac{\partial A_z}{\partial z} = 
        \sum_{i}^n \frac{\partial f(\vec{A} )}{\partial A_i}  \vec{e_i} 
    \]
    For a given divergence, we define the flux of a vector field. We define a surface with area 
    \(dx, dy\) with normal vector \(\hat{n}\). The flux differential is defined to be 
    \[
        d \phi  = \hat{n}  \cdot \vec{A} \mathrm{d}x  \mathrm{d}y    
    \]  
    For a large surface, the flux is defined to be 
    \[
        \phi  = \int d \phi  = \int  \vec{A}  \cdot \hat{n}  ds
    \]
        
    \end{definition}

    From here we get Gauss's divergence theorem (Green's Theorem)
    \[
        \oint  \vec{A}  \cdot \hat{n} ds = \int_V (\nabla \cdot \vec{A} ) dV 
    \]
    Given a uniform vector field, the flux is 0.
    Next we define the curl of the vector with is 
    \[
        {\nabla} \times \vec{A} 
    \]
    which defined how much the vector "swirls" around points. THe circulation of a field is denoted as 
    \[
        C= \oint \vec{A} \cdot d \vec{l}  
    \]
    We will get the between the curl and the circulation to be Stoke's Theorem
    \[
        \oint \vec{A} \cdot d\vec{l} = \int _s (\nabla \times A ) \cdot \vec{n} dS
    \]
\end{remark}

\begin{remark}
    Second Order Derivatives: scalar fields and vector fields

    Gradient \( \vec{\nabla } f\) which is vector. We can apply the del operator again with the dot product
    or the cross product which are two ways. For the divergence \(\vec{\nabla  } \cdot \vec{A}\) is 
    applying the del operator again \(\vec{\nabla } (\nabla A)\). For the curl, we have two combinations that are the same
    as the gradient. Thus, there are 5 second order derivatives\dots

    \begin{definition}
        \[
            \nabla  \cdot (\nabla f) = \sum_{i}^n \frac{\partial ^{2} }{\partial x_i ^{2}   } f = \nabla ^{2} f
        \]
        which is also defined as the laplacian
    \end{definition}

    \begin{definition}
        \[
            \nabla \times (\nabla  f) = (\frac{\partial ^{2} }{\partial y z} - \frac{\partial ^{2} }{\partial z,y} ) \hat{x}
            + (\frac{\partial ^{2} }{\partial z, x} - \frac{\partial ^{2} }{\partial x, z} ) \hat{y}
            + (\frac{\partial ^{2} }{\partial x,y} - \frac{\partial ^{2} }{\partial y,x} ) \hat{z} = 0
        \]
    \end{definition}
    \begin{definition}
        \[
            \nabla (\nabla  \cdot A) 
        \]
    \end{definition}
    \begin{definition}
        \[
            \nabla  \cdot (\nabla \times A ) = 0
        \]
    \end{definition}
    \begin{definition}
        \[
            \nabla \times (\nabla \times A) = \nabla (\nabla  \cdot A) - \nabla ^{2} A
        \]
    \end{definition}
\end{remark}

\begin{remark}
    Curvilinear Coordinates:
    
    There exists curvilinear coordinates which are useful in projection in orthoganal directions in 3D. 
    \begin{eg}
        The use of spherical and cylindrical coordinates are important for point charges, conducting wires, calculating the flux
        for certain special objects and planes, and other important objects. Spherical coordinates are useful in certain
        descriptions such as that of a point charge.
    \end{eg}

    \begin{definition}
        Spherical Coordinates:

        Suppose we had a point \(p\) in spherical coordinates with coordinates \((r,\phi ,\theta )\) 
        where the ranges are going to be 
        \[
            0 \leq  r \leq  \infty
        \]
        \[
            0 \leq  \phi  \leq  2\pi 
        \]
        \[
            0 \leq  \theta  \leq  \pi 
        \]
        where \(\phi \) represents the movement in the x-y plane and \(\theta \) represents rotation on the z-y plane. 
        The unit vectors are going to follow in the same planes. The conversion is as follows for cartesian to spherical

        \[
            x = r \cos \phi \sin \theta, 
            y = r \sin  \phi \sin \theta, 
            z = r \cos \theta 
        \]
        The infintisemal displacement will be 
        \[
            d\hat{l} = dr \hat{r} + r d\theta \hat{\theta } + r \sin  \phi d \phi \hat{\phi }
        \]
        For a small volume, we have 
        \[
            dV = r^{2}  \sin \theta  dr d \phi  d \theta , (dxdydz)
        \]
    \end{definition}

    \begin{definition}
        Cylindrical Coordinates \((r,\phi ,z)\) 
        \[
            0 \leq  r \leq  \infty , 
            0 \leq  \phi  \leq  2\pi , 
            - \infty \leq  z \leq  \infty
        \]
        intnfintisemal displacement:
        \[
             d\hat{l} = dr \hat{r} + dz \hat{z} + r d \phi  \hat{\phi }
        \]
        Volume:
        \[
            dV = r dr d \phi  dz
        \]
    \end{definition}
\end{remark}