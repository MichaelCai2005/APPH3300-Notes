\lecture{3}{Jan 29 10:10}{}
\begin{remark}
    Dirac delta function:

    Consider a vector field \( \vec{V} = \frac{1}{r^2} \hat{r} \). 
    The divergence of the field will result to 0 which is surprising 
    \[
        \vec{\nabla} \cdot \vec{v} = \vec{\nabla} \cdot \left( \frac{1}{r^2} \hat{r} \right) = \frac{1}{r^2} \frac{\partial }{\partial r} \left( r^2 \frac{1}{r^2} \right) 
        = 0
    \]

    For Gauss's Theorem ,we have that 
    \[
        \int  \vec{\nabla}  \cdot \vec{v} = \oint \vec{v} \cdot \hat{n} dS = 4\pi 
    \]
    This \(4\pi \) value holds for any radius, thus there exists a singularity which we can use to define 
    the dirac delta function. 
    \begin{definition}
        Properties:
        \[
            \delta (x) = 0, x \neq 0
        \]
        \[
            \int_{-\infty}^{\infty} \delta (x)  \,\mathrm{d}x  = 1 
        \]
        \[
            \int_{-\infty}^{\infty} \delta (x) f(x)  \,\mathrm{d}x = f(0) \int_{-\infty}^{\infty}  \delta (x) \,\mathrm{d}x  = f(0)
        \]
        \[
            \delta (x-a) = 0, x \neq a
        \]
        \[
            \int_{-\infty}^{\infty}  f(x) \delta (x-a)\,\mathrm{d}x  = f(a)
        \]
    \end{definition}
    There are ways to approximate \(\delta (x)\) by having a rectangular series ranging from \( \pm\frac{1}{2n}\) 
    \[
        \delta _n (x), n \hookrightarrow \infty 
    \]
    Another example is the function 
    \[
        \delta _n(x) = \frac{n}{\sqrt{\pi } } e^{-n^2 x^{2} } 
    \]
    where we must have 
    \[
        \lim_{n \to \infty} \delta _n(x) f(x) \mathrm{d}x  = f(0)
    \] and the normalization will will still have that 
    \[
        \int_{-\infty}^{\infty}  \delta (x)\,\mathrm{d}x   = 1
    \]

    \begin{definition}
        3D Dirac Delta function
        \[
            \delta ^{3} = 0 , \vec{x} \neq  0
        \]
        \[
            \int_{-\infty}^{\infty}  \delta (\vec{r} - \vec{r} _0)\,\mathrm{d}V = 0 
        \]
        \[
            \int_{-\infty}^{\infty} \delta ^3 (\vec{r} - \vec{r} _0 ) f(\vec{r} )\,\mathrm{d}V = f(\vec{r} _0)
        \]
    \end{definition}
    The physical case of a delta function is going to have the form 
    \[
         \vec{E} = \frac{q}{4 \pi \epsilon _0 } \frac{\hat{r} }{r^{2} }
    \]
    \[
        \vec{\nabla}  \cdot \vec{E}  = \frac{q}{4\pi \epsilon _0}\left( \vec{\nabla} \cdot \frac{\hat{r} }{r^{2} } \right) = \frac{q}{\epsilon _0} \delta  ^3 (\vec{r} )
    \]
    where we later find that \(q \delta ^3 (r) \) to be the charge density. Many times the derivative of the 
    delta function is not well defined. However, when we place it in a integral it is intuitive\dots
    \[
        \int_{-\infty}^{\infty} \mathrm{d}x f(x) \frac{\mathrm{d}\delta (x)}{\mathrm{d}x } = \frac{- \mathrm{d}f(x)}{\mathrm{d}x} \at{}{x=0}{}  
    \]
    
\end{remark}
\begin{theorem}
    Helmholt'z Theorem

    Given \( \vec{\nabla} \cdot \vec{F}  = D\) and \(\vec{\nabla} \times \vec{F}  = \vec{e} \) where the divergence of the curl is 0. 
    The theorem states that if \(D , \vec{c}  \hookrightarrow 0, \vec{r} \hookrightarrow \infty\)   and we have the condition 
    \(\vec{F}  \hookrightarrow  0, \vec{r}  \hookrightarrow  \infty \) . We can write \(\vec{F} \)  as two parts. 
    \[
         \vec{F} \equiv - \vec{\nabla} \phi  + \vec{\nabla}  \times \vec{A} 
    \] 
    where 
    \[
        \vec{\nabla} \cdot \vec{F}  = - \nabla ^{2}  \phi  = D
    \]
    \[
        \vec{\nabla} \times \vec{F}  = \nabla \times (\vec{\nabla}  \vec{A} ) = \vec{c} 
    \]
    If \(\vec{c} =0 \) , \( \vec{F} \)  is irrotational or curless and is written as \( \vec{F}  = - \vec{\nabla}  \phi \) 
    where \( \oint  \vec{F}  \cdot d \vec{l}  = 0 \) or also independent from the path. Another special case is when \(D= 0\) , we call the field 
    \( \vec{F} \)  to be divergenceless or solenoidal where we write \(\vec{F}  = \vec{\nabla} \times \vec{A} \implies  
    o\int  \vec{F} \cdot d \vec{S}  = 0\) which means that  it is independent of the surface for any given boundary. 

\end{theorem}

\begin{remark}
    Electrostatics: 

    Suppose we had a point charge \(q_0, q_1 \) at positions \( \vec{r} _0, \vec{r} _1 \)  where we can find that 
    \[
        \vec{F}_{01}  \frac{q_0 q_1}{4\pi \epsilon _0} \frac{\hat{r}_{01} }{|\vec{r}_{01}|^{2}}  , \vec{r}_{01}  = \vec{r} _0 - \vec{r} _1
    \]
    Suppose we have \(q_0\) fixed and move \(q_1\) in multiple places, we will see that the force 
    \[
        \vec{F}_{01}(\vec{r} _0) = q_0 \frac{q_1}{4\pi \epsilon _0} \frac{\hat{r}_0}{|\vec{r} ^{2} |}
    \]
    to be the field generated by charge \(q_1\) . Thus generally we have that 
    \[
        \vec{F}  = q \vec{E} 
    \]
    For a general set of particles we have
    \[
        \vec{E} (\vec{r} ) = \frac{1}{4 \pi  \epsilon _0} \sum_{j} \frac{q_j(\vec{r} - \vec{r} _j)}{|\vec{r}  - \vec{r} _j|^{3} }
    \]
    We can then generalize for a continuous charge distribution where we look at an infintesimal distrubtion of 
    \(dq = \rho dV\)  to be 
    \[
        \vec{E} (\vec{r} ) = \frac{1}{4\pi \epsilon _0} \int _V \frac{\rho (\vec{r} _1) (\vec{r} -\vec{r} _1)}{|\vec{r} -\vec{r} _1|^3} dV
    \]
    where the charge density \(\rho \) is in units of \(\frac{C}{m^3}\). We will later see for surface charges such as a charge sheet, 
    the charge is ditributed on the sheet with density \(\sigma \) , the infinitesimal charge  will be \(dq = \sigma  dA\) . 
    For line charges, we can write for a small segment to be \(dq = \lambda dl , \lambda   = \frac{C}{m}\) 
\end{remark}

\begin{eg}
Consider an \(\vec{E} \) field of a circular loop of charge along the \(z\)-axis. We calculate that 
\[
    dE_z = \frac{1}{4\pi \epsilon _0} \cdot \frac{\lambda dl}{R^{2} +z^{2} } \cdot \frac{z}{\sqrt{R^{2} +z^{2} } }
\]
\[
    E_z =  \frac{1}{4\pi \epsilon _0}  \frac{\lambda }{R^{2} +z^{2} } \cdot \frac{z}{\sqrt{R^{2} +z^{2} } } \int_0^{2\pi } dl
\]
\[
    = \frac{R}{2\epsilon _0}\frac{\lambda}{R^{2} +z^{2} } \cdot \frac{z}{\sqrt{R^{2} +z^{2} } }
\]

\end{eg}

\begin{eg}
Another example is calculating the electric field of an infinitely large charge sheet at a point \(z\) above the 
sheet. We can partition the sheet into a circle and see the symmetry that exists for rings. For a single ring we know th result. 
We can then integrate to get that 
\[
    E_z = \frac{z \sigma }{2\epsilon _0}\int_{0}^{\infty } \frac{R}{(R^{2} +z^{2} )^\frac{3}{2}} dR
\]
\[
    E_z = \frac{z \sigma }{2\epsilon _0} \frac{1}{\sqrt{R^{2} +z^{2} } } \at{}{0}{\infty } 
\]
\[
    E_z = \frac{\sigma}{2\epsilon _0}
\]
\end{eg}

\begin{remark}
    Gauss's Law
    \[
        \vec{E} (\vec{r} ) = \frac{q}{4\pi \epsilon _0} \frac{\hat{r} }{r^{2} }
    \]
    \[
        \oint _S \vec{E}  (\vec{r} ) \cdot \hat{n}  dS = \frac{q}{4\pi \epsilon _0} \frac{\hat{r} }{R^{2} } 4\pi R^{2}  = \frac{q}{\epsilon _0}
    \]
    We ge there that the total flux is going to be 
    \[
         Flux = \oint _S \vec{E}  \cdot \hat{n}  dS = \sum_{j} q_j /\epsilon _0
    \]
\end{remark}